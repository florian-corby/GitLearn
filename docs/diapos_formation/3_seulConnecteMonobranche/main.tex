\documentclass{beamer}
\usepackage{config}

%Information to be included in the title page:
\title[Git seul connecté mono-branche]{Git : mode d'emploi pour un usage seul, avec dépôt distant, sur une seule branche}
\author{Florian Legendre}
\institute{Université de Poitiers}
\date{Année 2020 - 2021}
\logo{\includegraphics[scale=0.1]{images/UP.png}}


%%% ============================================================= %%%
%%% ====================== Début des diapos ===================== %%%
%%% ============================================================= %%%

\begin{document}

\frame{\titlepage}

\begin{frame}
\frametitle{Table of Contents}
\tableofcontents[hideallsubsections]
\end{frame}


%% --------------------- %%
%%        SECTION        %%
%% --------------------- %%
\AtBeginSection[]
{
  \begin{frame}
    \frametitle{Table of Contents}
    \tableofcontents[sectionstyle=show/hide,subsectionstyle=show/show/hide]
  \end{frame}
}
\section{Configurer le suivi d'une branche distante}

% Subsection:
\subsection{Cloner un projet}

\begin{frame}[fragile]
\frametitle{Commande: git clone <url>}
\begin{mdframed}[style=Bash]
    \begin{lstlisting}[style=Bash, caption={Exemple de git clone}]
crex@crex:~/projects$ git clone https://github.com/torvalds/linux.git
Cloning into 'linux'...
remote: Enumerating objects: 7886755, done.
remote: Total 7886755 (delta 0), reused 0 (delta 0), pack-reused 7886755
Receiving objects: 100% (7886755/7886755), 2.98 GiB | 1.12 MiB/s, done.
Resolving deltas: 100% (6548657/6548657), done.
Updating files: 100% (71277/71277), done.
    \end{lstlisting}
    \end{mdframed}
\end{frame}

\subsection{Suivre une branche distante}
\begin{frame}
\frametitle{Commandes: git remote add <name> <url> / git remote rm <name>}
La première commande "git remote add" suivi d'un nom et d'une url permettent de suivre la branche distante désignée par l'url.\\
\smallskip

Le nom est un alias donné à l'url il permet notamment:
\begin{enumerate}
    \item De spécifier où faire le push d'une branche (Exemple: git push <nom> <nomDeMaBrancheLocale>)
    \item De spécifier quelle branche du dépôt distant on veut "pull" (Exemple: git pull <nom>)
    \item De supprimer un suivi de branche distante avec "git remote rm <name>"
    \item De spécifier une branche où on veut push/pull par défaut avec "git push $--$set-upstream <nom> <branch>"
\end{enumerate}
\end{frame}


%% --------------------- %%
%%        SECTION        %%
%% --------------------- %%
\AtBeginSection[]
{
  \begin{frame}
    \frametitle{Table of Contents}
    \tableofcontents[sectionstyle=show/hide,subsectionstyle=show/show/hide]
  \end{frame}
}
\section{Synchroniser son travail}

% Subsection:
\subsection{Récupérer les modifications d'un serveur distant}
\begin{frame}
\frametitle{Commandes: git fetch / git pull}

\begin{tabular}{ | m{13em} | m{13em} | }
    \hline
    
    \textbf{git fetch} & \textbf{git pull}\\
        
    \hline
    
    \begin{enumerate}
        \item Récupère seulement la dernière mise à jour de l'historique
        \item Il faut ensuite faire un git merge pour incorporer les changements
        \item Utile quand on veut anticiper des conflits ou qu'on n'a pas tout à fait fini son travail
    \end{enumerate}
    & 
    
    \begin{enumerate}
    \item Récupère seulement la dernière mise à jour de l'historique et incorpore les changements simultanément
    \item Elle est plus utilisée que git fetch
    \end{enumerate} \\
    
    \hline
\end{tabular}

\end{frame}


% Subsection:
\subsection{Ajouter ses mises à jour sur un serveur distant}
\begin{frame}
\frametitle{Commande: git push}
La commande "git push" permet de mettre à jour la branche distante en incorporant dans la branche distante l'historique et les fichiers locaux.
\bigskip

\textbf{ATTENTION:} 
\begin{enumerate}
    \item Il faut avoir fait un git pull avant de l'utiliser
    \item Il faut avoir les droits
\end{enumerate}
\end{frame}


\end{document}